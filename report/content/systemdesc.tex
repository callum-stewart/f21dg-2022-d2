\section{System Description}

This section contains documentation detailing the top-level design of the system, alongside further documentation on how the future developer reading this
can modify and reuse this code in the future. It will first guide through an overview of how the entire system works, 
building up background information that should be understood first, then continuing on to dive into specifics.

\subsection{Design overview}

\subsubsection{Pyodide.}

That word may not mean much, however it is critical to how this whole project fits together. 
Pyodide is a full CPython reference implementation 
compiled and embedded entirely in browser using a virtual instruction set language named WebAssembly. Using WebAssembly, it is possible to get near (system, configuration, and more dependant)
native run-times of languages while encapsulating software systems in browser tabs. Even while being a relatively new 
technology - initially appearing in 2017 \cite{10.1145/3062341.3062363} - it 
is supported in the most commonly used browsers, including mobile OS browsers. 

Not only does this allow custom Python code that can be run directly in browser, but we can also make
use of Pythons humongous pre-existing library of pre-built packages (wheels) that are written in pure python. 

Note: Some packages do however contain C code alongside their Python code, but using the same system that enabled the CPython implementation to be compiled to WebAssembly,
we can do the same with the Python packages containing C, allowing almost any package to run entirely in browser.

In summary, Pyodide enables the use of pre-existing, tested, and verified packages for signal generation, EMD and STFT analysis, and graph generation, combining them carefully 
- alongside performing personalised unit and integration testing - into the software system. 
However Pyodide is not all we shall need to produce the full working system, and there are many caveats with Pyodide itself.


\subsubsection{How we make use of Pyodide.}\label[]{howweusepyodide}

Pyodide appears to be an all-powerful tool, however how much use can we really get out of it? While it is able to solve many problems
revolving around complex algorithm implementation, signal generation and combining, and graph generation, we still need methods using standard web technologies, i.e. HTML, CSS, JavaScript,
to make the system. It is these cross-overs between Pyodide technologies and standard web technologies that created the majority of challenges during implementation.

We made use of several Python packages during implementation to varying degrees, these packages are listed below:

\begin{itemize}\label[]{requiredpackages}
    \item {\bf NumPy - } NumPy is the fundamental Python package for scientific computing, implementing multidimensional arrays, matrix structures, alongside basic FFT's and signal generation techniques.
    We cover the specifics of which functions that we use in the \hyperref[signalgeneration]{Signal Generation}, \hyperref[emdsection]{EMD}, and \hyperref[stftsection]{STFT}.
    sections.

    \item {\bf SciPy - } SciPy is a scientific Python package built on-top of NumPy, incorporating further functions for signal processing, optimisation, and statistics. We make use of several available
    functions when \hyperref[signalgeneration]{Generating Signals}, alongside performing \hyperref[stftsection]{STFT Analysis}.

    \item {\bf mpld3 - }\label[]{mpld3} A Python package that exports a graph MatPlotLib plot to HTML, CSS, JavaScript, and most importantly D3. D3 is a powerful graphics 
    library for JavaScript, that allows for interactive graphs entirely in browser. More specifically, mpld3 makes use of a the MatPlotLib mplexporter framework, which parses an 
    input MatPlotLib plot, and outputs a JSON representation. This representation can then be read in by mpld3.js, a standalone JS framework built on-top of D3, which exports the HTML graph.

    \item {\bf micropip - } A Python package specifically for Pyodide, this package allows loading Python packages from PyPi or any arbitrary source at runtime. 

    \item {\bf emd/pyhht - }
    
    \item {\bf Pandas - }

\end{itemize}

The use of Pyodide also has impacts on how the rest of the system is shaped, and choices were made during development on how we would work around caveats that came to light during 
cross technology implementation. These caveats are detailed below, and may be referenced later in the manual:

\begin{itemize}
    \item {\bf Web Workers - } Using Pyodide with its default implementation runs all code in the main tab render thread. This can present a serious
    problem in any system that makes use of graphical interface, as during any long length computation, the tab can become completely unresponsive, harming the user experience.
    To fix this, we made use of Pyodide only inside Web Workers. These Web Workers are a simple means to run scripts in background threads, and therefor allow Python scripts to run
    in a completely separate thread, leaving the tab render thread to continue unimpaired. 
    There are however consequences of this fix. When Pyodide is constructed in a Web Worker, it creates its entire own interpreter. This means we cannot
    make use of the Python global namespace for information handling or sharing, and must share messages from Web Workers back to the Javascript storage handler via post messages.
    Further complicating the matter, there are restrictions imposed by both Pyodide and Javascript on what objects can be passed and auto-translated between environments.
    We are able to alleviate this problem though, by passing around Javascript objects that execute the Python code and only returning the final result - using the bullet point below - 
    a raw HTML graph, with all included controls and data, at no point does there require any Python $ \to $ Javascript object translation.
    A final problem is that the page DOM nor session storage can be accessed from inside a Web Worker Pyodide interpreter, and any attempt to access parts of the `js' package causes Pyodide to crash.
    However as we return values from the Web Workers, there is no need to access the DOM directly, and this does not present a problem.

    \item {\bf MatPlotLib - } MatPlotLib is a comprehensive Python package used to create various forms of visualisation. In this case, and as detailed later in the 
    \hyperref[stftsection]{specific details on STFT implementation} and \hyperref[emdsection]{specific details on EMD implementation}, we make use of the spectrogram
    and line graph visualisations contained within the library. However, when using the Web Worker technology detailed above, MatPlotLib makes an attempt to access part of the `js' package
    as part of its loading process. This immediately crashes the Web Worker thread, even when we don't want MatPlotLib to access the DOM directly, just create a graph to be stored.
    It can however, be fixed. MatPlotLib makes use of a frontend $ \leftrightarrow $ backend system,  with the frontend handling
    the user facing code, and the backend handling the heavy lifting behind-the-scenes to make the figure. MatPlotLib also automatically selects which backend to use depending on the available
    system, and we found that is part of the default backend code that is attempting to access the `js' package. 
    To stop this, we can force MatPlotLib to use a different backend, such as AGG (exports high quality images using the Anti-Grain Geometry engine), enabling graphs to be saved as required.

    \item {\bf mpld3 - } While MatPlotLib can generate graph images and interactive graphs when using a framework such as Qt, without these frameworks it cannot alone export interactive graphs
    like those that we need to display the signals. We use mpld3 to generate these interactive graphs from inside Python. However, the same issue was encountered with this library as detailed above in the
    MatPlotLib package - it was trying to access the `js' package inside a Web Worker. Once again to solve this, we changed the backend that mpld3 used, which prevented the problem.

    \item {\bf Bookmarking - } As mentioned previously, due to the use of Web Workers, the Python global name space for storing Python generated signal types is not available. However, 
    we can work around this by using the browsers local storage as a replacement, simply involving writing JavaScript handling code.
\end{itemize}


\subsubsection{How Pyodide fits in with the rest of the system.}

Hopefully now through the previous sections it is clear that while Pyodide has some hurdles that must be jumped, it provides an excellent staging ground for this system to launch from.
It must now integrate with the rest of the client side web stack, being executed for example through JavaScript events from button presses, or returning data through post messages to be added
to the page DOM.

To keep things simple, we have a single Web Worker running Pyodide inside, setup importing the required packages, and implementing any fixes to the caveats in the system as detailed \hyperref[howweusepyodide]{above}.
This single Web Worker is setup to accept and return post messages, a method of communication between threads in JavaScript. An initial JSON post message is sent to the Worker with signal information (stored
in session storage) and analysis type. This JSON message is then forwarded into the Pyodide code, where Python decides which method of analysis should be performed, generates the signals from input parameters,
combines said signals, and finally performs the analysis. The result of all this Python code is a string of HTML code, that is returned once more through a post message from the Web Worker to the main thread,
where it is injected into the DOM.

This remaining web stack makes use of Bootstrap material design to provide a consistent and modern design to the front end, detailed further in the \hyperref[frontendsection]{Front-end interface section}.
Pyodide graphs are then generated in HTML and injected directly into the DOM, making use of the \hyperref[mpld3]{mpld3 package}. Alongside this, the whole system only uses plain JavaScript for simplicities sake.

\subsection{Front-end Interface}\label[]{frontendsection}

The front-end interface is built so that any user can experiment and perform EMD and STFT analysis both on their own uploaded data and on data that they have generated through this system.

The bulk of the front-end is contained with in the \verb|src/index.html| file. Inside the \verb|<head>| of the document, the styled version of bootstrap is loaded, alongside our bundled \verb|main.bundle.js| 
JavaScript code. The system makes use of webpack to bundle all JavaScript code into a single file, alongside optimising and removing any duplication. In the remainder of the \verb|<body>| of the document, 
bootstrap's buttons and navigation constructs are used to create the navigation bar, along with option buttons. Further in the document, the modal containing information on the types of signal that 
the user can generate is created. This is created using an accordion class from bootstrap, allowing for a drop-down menu for each signal type. Inside these signal types, details on the available
input parameters are also listed, providing clear information to the user on what their choice of input signals will achieve. 

Moving on to the \verb|main.js| file, the first step is loading in all JavaScript modules and their relevant functions into the \verb|main.js| file. Using this module and loading system means modularity can
be implemented during development, making code easier to understand, test, and refactor, while using \verb|webpack| as mentioned previously to bundle our code together when it is served to the end user.
Next, document \verb|querySelector|'s are used to find, and then attach click event listeners to the required user-interface buttons. Using this method means as much arbitrary code as required can be
executed upon button press, such as saving bookmarks or setting the signal analysis method.
Alongside this, the \verb|DOMContentLoaded| event is attached to. This event is only called when the page has completed loading, and means that it is possible to manipulate the DOM in any way required,
creating a \verb|promise| that the page is ready.
Once this promise is fulfilled, previously mentioned module classes are instantiated, such as \verb|InfoPanel|. These classes expose functions that generate or modify content in the DOM. In the example 
of \verb|InfoPanel|, the \verb|displayInfoPanel| function is exposed, taking a boolean argument on whether it should be displayed or not. This class and function also grab data from a separate JSON
file named \verb|methodInfo.json|. When required to be displayed, the relevant information from the \verb|methodInfo.json| file is displayed to the user, providing advantages, disadvantages, and a description
of each signal analysis type.

\subsubsection{Input Selection}
The input selection is shown as two different tabs on the front end namely 'Upload Signal' and 'Create Signal'. These two different tabs are implemented through modules \verb|UploadSignal.js| and \verb|ConfigureSignal.js|. 
Uploading signals is mainly covered through the fileIO module for file processing and passing through the signal data as the front-end simply includes the file input element and the button to generate the 
graphs as there are no other configuration settings involved with this method other than choosing the analysis method.
Configuring signals within the application is implemented through dynamic forms created for each of the signal types as they all required different parameters to create. 
These are loaded in through HTML templates using the ES6 syntax and create a button or 'chip' for every signal produced so they can be edited and deleted as necessary. 
These actions all update the URL as required to ensure it is accurate when the user bookmarks the application state. All the form text inputs have been limited to numerical values though the required attribute has not been able to be implemented as there were issues with Bootstrap custom forms and using browser defaults. 
There is also a help link on the signal form to give more information on the signal types units. The combination method selector includes a tool-tip to explain that combining via product should be limited to certain data types as it was decided that we should not restrict the user experimentation as much as possible. 

\subsubsection{Bookmarking}
As part of our original specification we were tasked to create a method of saving and sharing the users application settings and signal configurations. The implemented solution is based off creating a bookmark 
feature that copies the current URL to the user's clipboard. Bookmarking is handled within the \verb|bookmark.js| module and is used to update the URL parameters with the users settings and choices through the UI.
 In order to consistently update the URL when an analysis method is chosen or signal parameter is added or edited the URL is updated on each button click which triggers a UI change for that action as well as saving the URL string within session storage. 
 This string can then be processed to create an object that can be used as the input data for the graphs. There are different methods for dealing with the signal parameters as the keys follow a format of \verb|[signal number]-[signal parameter name]=[signal value]| in order to differentiate and identify the different signal's values within the URL.

\subsection{File IO}

\subsubsection{Main Logic}
The application supports users supplying their own input data for both EMD and STFT analysis. 

This functionality is achieved in a separate class in the source code located at {\bf src/fileIO.js}. 

The constants required by the class, and those that reference it, are declared at the beginning. These include an array of supported file types (this is currently just {\bf .csv}, but utilizing an array will make it easier in the future to add more supported file-types) and a {\bf csvDataKey}, which is used as the key to store uploaded csv data in the browser's sessionStorage.

In typical use-cases, the class is invoked via the {\bf parseFile()} method, which will attempt to read a .csv file selected by the user. There is a safety check to verify that a file has indeed been selected by the user, and the {\it 'Generate Graphs'} button hasn't just been pressed without something to process: if that is the case, the method will safely return early, without an error.

After verifying that a file has indeed been selected by the user, there is another safety check to validate that the type of file selected is supported; this is in addition to specifying this as a parameter of the HTML input option in {\bf src/modules/uploadSignal.js}. If more file types are to be supported in the future, this parameter should reference the static {\it supportedFileTypes} array mentioned earlier to reduce tediousness of the work, as only one file will need to be altered to allow for more file extensions to be accepted.
If, for some reason, a file-type has slipped through the cracks and this check fails, an alert will be displayed informing the user that the given file-type is not supported, and to try again. The method will then return early.

Finally, just before the file is read in, there is a check to determine whether {\it csvDataKey} in browser storage is currently populated; if it is, then it means it's data from a previous upload and should be cleared before progressing. 

Afterwards, a new {\bf FileReader()} object is created to read in the supplied file. Error/Progress/Abort/Load handlers are all attached to the object to ensure robustness. Once the {\it 'loadend'} event is triggered, a call is made to the {\bf loaded(target) function}. This is the function responsible for actually parsing the data read in by the file reader.

Upon being called, the {\it loadend} method will change the status displayed to the user as "Finished Loading!" and will then call the {\bf csvToArray} method to convert the text which has just been read in to a JavaScript array. Once finished being converted, the resultant array is then converted into a JSON string and stored in sessionStorage under the {\it csvDataKey} key. The object is converted to a JSON String to preserve the structure of the data (without this, the '[' and ']' are lost, making the data much harder to parse) and for consistency with other classes utilizing sessionStorage.

The {\bf csvToArray} method invokes an external library called {\it convertCSVToArray} which handles all the transformation logic. The headers are stripped, the output type is supplied as 'array' and the separator is designated as ',', and the library handles the rest. Once the data is read in, there is a check to ensure that only numerical data exists in the resultant array via the {\bf onlyNumericalDataInCSVArray(arr)} method.

The {\bf onlyNumericalDataInCSVArray} method iterates through the entire array to ensure that there is no non-numerical data contained in the array (this is to make things easier for the Python code down the line). The only exception to this is if the code represents 'e', 'E' or '-' in place of {\it exponent} values. If this check fails, the method returns false and an error is thrown in the parent {\bf csvToArray} method. Otherwise, the method returns true. 

After this process is completed, the data contained in the supplied file is available to be read by any class that requires it via the {\it csvDataKey} key in the browser's sessionStorage. 

\subsubsection{Helper Class}
A small class was created to assist with reading and writing to the Browser's sessionStorage. This class is located in {\bf src/helpers/sessionStorage.js}.

A disclaimer is at the top of the class notifying which browser versions support session storage (and thus the functionality to upload files). The remainder of the class is then populated with getter/setter methods as well as a method to check if a value is associated with a given key. 

\subsubsection{CSV File Format}
This section provides guidance for how to structure your .csv file(s) so that they can be read by the application. 

Although the title of your headers doesn't particularly matter, their number and ordering does. The application expects your headers to follow the convention {\bf Time, Value}, such that the first value represents the time of the event (your X axis value), and the second represents the value at the given time (your Y axis value). 

An example is provided below: 
\begin{table}[!ht]
    \centering
    \begin{tabular}{|l|l|}
    \hline
        {\bf Time} & {\bf Value} \\ \hline
        0.0 & 0.0 \\ \hline
        0.00628947 & 0.00012583 \\ \hline
        0.01257895 & 0.00050233 \\ \hline
    \end{tabular}
\end{table}


\subsection{Generating Graphs}\label[]{gengraphs}

The process of generating graphs, generating signals, combining signals, and performing analysis, is all performed in one large swoop, involving heavy use of Pyodide and Python,
and is subsequently outlined in the following section.

Prior to the journey beginning, when the page is initially constructed an instance of the Pyodide Web Worker must first be created. This Web Worker creates a Pyodide Python interpreter inside it, and installs the 
\hyperref[requiredpackages]{Required Packages}, using both micropip and the \verb|pyodide.loadPackage| methods. Once these packages have been loaded, and the Worker is satisfied that it is ready to receive data, it assigns a 
positive value to the \verb|pyodideReady| boolean, signalling to the rest of the code that the Worker is ready.

The Web Worker then sits idle, until the pressing of the \verb|Generate Graphs| button present in the \hyperref[frontendsection]{User Interface}. This button press calls an event, which in turn makes a chain of function calls.
First, it calls the \verb|displayLoadingGraphs| function contained inside the \verb|graphs.js| file. 
This function creates the placeholders for the graphs in the DOM, and adds a loading animation to the placeholders to tell the user that there is some background processing occurring.
Next, the \verb|handleCallPyodide| function from \verb|pyodideHelpers.js| is called, which creates in turns calls the \verb|evaluatePython| function, while creating a promise on said function to return some data for
future processing.
This \verb|evaluatePython| function creates a promise, which when resolved correctly, sends a post message to the previously mentioned Pyodide Web Worker containing the method that the data was created (uploaded or generated 
using the system), the analysis method to be performed, and if the signal was generated using the system, the information about the signal. The Web Worker then takes these input parameters, and decides which functions need to be called.

If the signal provided was generated using the tool, then the \hyperref[signalgeneration]{Signal Generation} code shall be called, otherwise, the program will skip straight to the \hyperref[stftsection]{STFT} or \hyperref[emdsection]{EMD}
code.

\subsubsection{Signal Generation}\label[]{signalgeneration}

Signal generation is handled entirely in Python, inside the Pyodide Web Worker's \verb|script.py| file, with the signal data from the user passed in via the \verb|analysis_runner| function call. This function then parses the input data, and generates
signals according to the input parameters from the user, calling the relevant functions inside the \verb|signals.py| file. Inside this file, a separate function, e.g. \verb|simple_sin|, \verb|chirp|, \verb|white_noise|, is created, that 
accepts the required input parameters, and returns a time series of points over a predefined period. These functions also make use of the \hyperref[requiredpackages]{Required Packages}, specifically NumPy and SciPy, to generate
the time-series, alongside Pandas for series storage.

\subsubsection{Signal Combining}

\subsubsection{STFT Analysis}\label[]{stftsection}


\subsubsection{EMD Analysis}\label[]{emdsection}



