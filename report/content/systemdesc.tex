\section{System Description}

This section contains documentation detailing the top-level design of the system, alongside further documentation on how the future developer reading this
can modify and reuse this code in the future. It will first guide through an overview of how the entire system works, 
building up background information that should be understood first, then continuing on to dive into specifics.

\subsection{Design overview}

\subsubsection{Pyodide.}

That word may not mean much, however it is critical to how this whole project fits together. 
Pyodide is a full CPython reference implementation 
compiled and embedded entirely in browser using a virtual instruction set language named WebAssembly. Using WebAssembly, it is possible to get near (system, configuration, and more dependant)
native run-times of languages while encapsulating software systems in browser tabs. Even while being a relatively new 
technology - initially appearing in 2017 \cite{10.1145/3062341.3062363} - it 
is supported in the most commonly used browsers, including mobile OS browsers. 

Not only does this allow us to write our own Python code that can be run in browser, but we can also make
use of Pythons humongous pre-existing library of pre-built packages (wheels) that are written in pure python. 

Note: Some packages do however contain C code alongside their Python code, but using the same system that enabled the CPython implementation to be compiled to WebAssembly,
we can do the same with the Python packages containing C, allowing almost any package to run entirely in browser.

In summary, Pyodide enables us to use pre-existing, tested, and verified packages for signal generation, EMD and STFT analysis, and graph generation, combining them carefully 
- alongside performing our own unit and integration testing - into our software system. 
However Pyodide is not all we shall need to produce the full working system, and there are many caveats with Pyodide itself.


\subsubsection{How we make use of Pyodide.}

Pyodide appears to be an all-powerful tool, however how much use can we really get out of it? While it is able to solve many of our problems
revolving around complex algorithm implementation, signal generation and combining, and graph generation, we still need methods using standard web technologies, i.e. HTML, CSS, JavaScript,
to make the system. It is these cross-overs between Pyodide technologies and standard web technologies that presented us with the majority of challenges during implementation.

As such, our use of Pyodide has impacts on how we shape the rest of the system, and choices were made during development on how we would work around caveats that came to light during 
cross technology implementation. These caveats are detailed below in a broad manner, and are again explored in specific sections later in this manual:
\newpage
\begin{itemize}
    \item {\bf Web Workers - } Using Pyodide with its default implementation runs all code in the main tab render thread. This can present a serious
    problem in any system that makes use of graphical interface, as during any long length computation, the tab can become completely unresponsive, harming the user experience.
    To fix this, we made use of Pyodide only inside Web Workers. These Web Workers are a simple means to run scripts in background threads, and therefor allow us to run our
    Python scripts in a completely separate thread, leaving the tab render thread to continue unimpaired. 
    There are however consequences of this fix. When Pyodide is constructed in each Web Worker, it creates its entire own interpreter. This means we cannot
    make use of the Python global namespace for information handling or sharing, and must share messages from Web Workers back to the Javascript storage handler via post messages.
    Further complicating the matter, there are restrictions imposed by both Pyodide and Javascript on what objects can be passed and auto-translated between environments.
    We are able to alleviate this problem though, by passing around Javascript objects that execute the Python code and only returning the final result - using the bullet point below - 
    a raw HTML graph, with all included controls and data, at no point does there require any Python $ \to $ Javascript object translation.
    A final problem is that the page DOM cannot be accessed from inside a Web Worker Pyodide interpreter, and any attempt to access parts of the `js' package causes Pyodide to crash.
    However as we return values from the Web Workers, there is no need to access the DOM directly, and this does not present a problem.

    \item {\bf MatPlotLib - } MatPlotLib is a comprehensive Python package used to create various forms of visualisation. In our case, and as detailed later in the 
    \hyperref[stftsection]{specific details on our STFT implementation} and \hyperref[emdsection]{specific details on our EMD implementation}, we make use of the spectrogram
    and line graph visualisations contained within the library. However, when using the Web Worker technology detailed above, MatPlotLib makes an attempt to access part of the `js' package
    as part of its loading process. This immediately crashes the Web Worker thread, even when we don't want MatPlotLib to access the DOM directly, just create a graph to be stored as a 
    Base64 string. It can however, be fixed. MatPlotLib makes use of a frontend - backend system,  with the frontend handling
    the user facing code, and the backend handling the heavy lifting behind-the-scenes to make the figure. MatPlotLib also automatically selects which backend to use depending on the available
    system, and we found that is part of the default backend code that is attempting to access the `js' package. 
    To stop this, we can force MatPlotLib to use a different backend, such as AGG (exports high quality images using the Anti-Grain Geometry engine), enabling us to save graphs as we require.

    \item {\bf mpld3 - }
    \item {\bf Bookmarking - } As mentioned previously, due to the use of Web Workers, the Python global name space for storing Python generated signal types is not available. However, 
    we can work around this by using the browsers local storage as a replacement, simply involving writing JavaScript handling code.
\end{itemize}

\subsubsection{How Pyodide fits in with the rest of the system.}




\subsection{Front-end Interface}

\subsubsection{Graphing}


\subsubsection{Bookmarking}


\subsection{Signal Generation}

\subsubsection{Signal Combining}

\subsection{File IO}


\subsection{STFT Analysis}\label[]{stftsection}


\subsection{EMD Analysis}\label[]{emdsection}



