\section{Preface}

\subsection{How to use this manual}

If you are reading this manual, congratulations, you own your very own copy of the F21DG group project implementing EMD and STFT analysis entirely in browser! 
This manual contains information detailing both how the end user will make use of the system, and how future software development students may make use of the system.

Alongside this, the manual attempts not only to share details on how the system works, but guide future developers through why certain decisions were made, 
so that if in the future they wish to modify it, we may already have an explanation on why parts are designed the way that they are. 

We do hope that this system can be expanded, and that by the time students do wish to do so, further advances in software that we began to look at (e.g. WebGL)
can be fully utilised to further improve the teaching abilities of this software.

\subsection{Description of the user}

This created tool intents to be an educational resource allowing end users to interactively explore the differences between STFT and EMD time-series analysis techniques.
However, while the tool is a prototype, we do wish for it to have a high degree of usability. As such, the tool is implemented in such a way that users have the 
ability to bookmark examples, use pre-generated signals as inputs, view the demonstration of their combined input signal being decomposed, and much more - detailed in 
the \hyperref[systemdesc]{System Description} section.

\newpage